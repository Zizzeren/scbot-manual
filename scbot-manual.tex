\section{SCBot Commands \& Usage}

\subsection{General Usage}

\subsubsection{Discord Terms}
\paragraph{IDs}
\label{discord:message-id}

\subsection{List of Commands}

\subsubsection{Alias}
\begin{description}
	\item[\command{alias}]\label{com:alias}\index{alias}
		The alias command is used to create aliases, which are shortcuts to other commands. For example, rather than typing \command{alias add}, it could be shortened to \command{aa}.
		
	\begin{description}
		\item[\command{alias add $<$command$>$ $<$to\_execute$>$}]\label{com:alias-add}
			Adds an alias to a command. Example: \command{alias add aa alias add}
			
		\item[\command{alias del $<$command$>$}]\label{com:alias-del}
			Deletes an alias. Example: \command{alias del aa}
			
		\item[\command{alias help $<$command$>$}]\label{com:alias-help}
			Attempts to give help for the aliased command. Example: \command{alias help aa} should return the help for \command{alias add}.
			
		\item[\command{alias list}]\label{com:alias-list}
			Lists all the aliases in the server. Sends to DMs.
			
		\item[\command{alias show $<$command$>$}]\label{com:alias-show}
			Shows what an alias executes. Example: \command{alias show aa} responds with \texttt{alias add}.
	\end{description}
\end{description}

\subsubsection{Ascii}
\begin{description}
	\item[\command{ascii $<$text$>$}]\label{com:ascii}\index{ascii}
		Expands text into block ASCII form.
\end{description}

\subsubsection{AutoInfo}

\subsubsection{Autorole}

\subsubsection{Avatar}

\subsubsection{Bigmoji}

\subsubsection{Blizzard}

\subsubsection{ChannelLogger}

\subsubsection{ConfigStore}

\subsubsection{Dice}

\subsubsection{Drawing}

\subsubsection{Economy}

\subsubsection{Filterban}

\subsubsection{General}

\subsubsection{Giveme}

\subsubsection{Kickrole}

\subsubsection{Mod}
\begin{description}
	
	\item[\command{cleanup}]\label{com:cleanup}\index{cleanup}
		Deletes a number of messages, depending on the argument you give it.
		
	\begin{description}
		
		\item[\command{cleanup messages $<$number$>$}]\label{com:cleanup-messages}
			Deletes the number of messages you specify, plus the message you sent to trigger the command.
		\item[\command{cleanup after $<$id$>$}]\label{com:cleanup-after}
			Deletes all messages after a specified message, using the \hyperref[discord:message-id]{message ID} to refer to a message.
			
	\end{description}

	\item[\command{filter}]\label{com:filter}\index{cleanup}
		Allows you to filter messages that are sent with certain banned words, deleting them instantly.
		
\end{description}

\subsubsection{Modlog}

\subsubsection{NeedsMoreJpeg}

\subsubsection{Pic}

\subsubsection{RCG}

\subsubsection{Ratewaifu}

\subsubsection{ReactPoll}

\subsubsection{Register}

\subsubsection{RemindMe}

\subsubsection{Scheduler}

\subsubsection{Spoiler}

\subsubsection{Steam}

\subsubsection{Stickyroles}

\subsubsection{Streams}

\subsubsection{Trivia}

\subsubsection{Welcome}

\subsubsection{Say}

% Etc - one subsubsection for each cog